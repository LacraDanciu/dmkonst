\documentclass[../report.tex]{subfiles}
\begin{document}
This report explains the implementation of a multi-cycle processor in VHDL. The processor was designed with a simplified version of Microprocessor without Interlocked Pipeline Stages (MIPS) instruction set. 

\section{Requirements}

The main requirement for the design of the processor was that it should use MIPS instruction set with an encoding format described in the compendium appendix \cite[p. 64]{compendium}. The processor should have instructions for each of the classes described in Table \ref{t:req}.

\begin{table}[H]
     \caption {Requirements}
	\label{t:req}
    \centering
    \begin{tabular} {l l}
       	 \bf Arithmetic logic unit (ALU): & ADD, SUB, SLT, AND, OR instructions \\
       	 \bf Branch: & Conditional branch instruction \\
	     \bf Memory: & LOAD and STORE instructions  \\
	     \bf Load Immediate (LDI): & load the register with the given value \\
	     \bf Jump instruction: & J-jump to the specified address \\

    \end{tabular}
\end{table}


The design was also required testing by several test benches and then run on an FPGA. The requirements are described in full details in the compendium \cite[p. 44]{compendium}

\end{document}